\chapter{Kurzfassung}

Für die Realisierung von komplexen und skalierbaren Softwaresystemen ist ein Trend zur sogenannten Microservice"=Architektur, anstelle von monolithischen Architekturen zu beobachten. Bei diesem Architekturmuster handelt es sich um eine spezielle serviceorientierte Architektur, die aus tatsächlichen Anwendungsfällen von großen Internetfirmen hervorgegangen ist. In dieser Arbeit werden die Gründe für diesen Paradigmenwechsel gesucht und erläutert. Außerdem werden verschiedene Technologien für die erfolgreiche Umsetzung einer Microservice"=Architektur untersucht und miteinander in Verbindung gesetzt. Zu diesen Technologien zählen die Virtualisierung mit Containern, serverlose ereignisgesteuerte Funktionen und das Aktorenmodell. Auf Basis der genannten Konzepte gibt diese Arbeit Empfehlungen, für welche Szenarien der Microservice"=Ansatz überhaupt geeignet ist und wie dieser dann schlussendlich effektiv umgesetzt werden kann.